% Options for packages loaded elsewhere
\PassOptionsToPackage{unicode}{hyperref}
\PassOptionsToPackage{hyphens}{url}
%
\documentclass[
  11pt,
  ignorenonframetext,
]{beamer}
\usepackage{pgfpages}
\setbeamertemplate{caption}[numbered]
\setbeamertemplate{caption label separator}{: }
\setbeamercolor{caption name}{fg=normal text.fg}
\beamertemplatenavigationsymbolsempty
% Prevent slide breaks in the middle of a paragraph
\widowpenalties 1 10000
\raggedbottom
\setbeamertemplate{part page}{
  \centering
  \begin{beamercolorbox}[sep=16pt,center]{part title}
    \usebeamerfont{part title}\insertpart\par
  \end{beamercolorbox}
}
\setbeamertemplate{section page}{
  \centering
  \begin{beamercolorbox}[sep=12pt,center]{part title}
    \usebeamerfont{section title}\insertsection\par
  \end{beamercolorbox}
}
\setbeamertemplate{subsection page}{
  \centering
  \begin{beamercolorbox}[sep=8pt,center]{part title}
    \usebeamerfont{subsection title}\insertsubsection\par
  \end{beamercolorbox}
}
\AtBeginPart{
  \frame{\partpage}
}
\AtBeginSection{
  \ifbibliography
  \else
    \frame{\sectionpage}
  \fi
}
\AtBeginSubsection{
  \frame{\subsectionpage}
}
\usepackage{lmodern}
\usepackage{amssymb,amsmath}
\usepackage{ifxetex,ifluatex}
\ifnum 0\ifxetex 1\fi\ifluatex 1\fi=0 % if pdftex
  \usepackage[T1]{fontenc}
  \usepackage[utf8]{inputenc}
  \usepackage{textcomp} % provide euro and other symbols
\else % if luatex or xetex
  \usepackage{unicode-math}
  \defaultfontfeatures{Scale=MatchLowercase}
  \defaultfontfeatures[\rmfamily]{Ligatures=TeX,Scale=1}
  \setmainfont[]{Linux Libertine O}
  \setmonofont[]{DejaVu Sans Mono}
\fi
\usetheme[]{clean}
\usefonttheme{serif} % use mainfont rather than sansfont for slide text
% Use upquote if available, for straight quotes in verbatim environments
\IfFileExists{upquote.sty}{\usepackage{upquote}}{}
\IfFileExists{microtype.sty}{% use microtype if available
  \usepackage[]{microtype}
  \UseMicrotypeSet[protrusion]{basicmath} % disable protrusion for tt fonts
}{}
\makeatletter
\@ifundefined{KOMAClassName}{% if non-KOMA class
  \IfFileExists{parskip.sty}{%
    \usepackage{parskip}
  }{% else
    \setlength{\parindent}{0pt}
    \setlength{\parskip}{6pt plus 2pt minus 1pt}}
}{% if KOMA class
  \KOMAoptions{parskip=half}}
\makeatother
\usepackage{xcolor}
\IfFileExists{xurl.sty}{\usepackage{xurl}}{} % add URL line breaks if available
\IfFileExists{bookmark.sty}{\usepackage{bookmark}}{\usepackage{hyperref}}
\hypersetup{
  pdftitle={CS 1340:Fall 2020:Lecture 02},
  pdfauthor={Mark Fontenot, PhD},
  hidelinks,
  pdfcreator={LaTeX via pandoc}}
\urlstyle{same} % disable monospaced font for URLs
\newif\ifbibliography
\usepackage{longtable,booktabs}
\usepackage{caption}
% Make caption package work with longtable
\makeatletter
\def\fnum@table{\tablename~\thetable}
\makeatother
\setlength{\emergencystretch}{3em} % prevent overfull lines
\providecommand{\tightlist}{%
  \setlength{\itemsep}{0pt}\setlength{\parskip}{0pt}}
\setcounter{secnumdepth}{-\maxdimen} % remove section numbering
% \usepackage{libertine}
% \usepackage[T1]{fontenc}

\usepackage{enumitem}
\usepackage{amsfonts}
% level one
\setlist[itemize,1]{label=$\bullet$}
% level two
\setlist[itemize,2]{label=$\circ$}
% level three
\setlist[itemize,3]{label=$\star$}

% \usepackage{fontspec}
% \setromanfont[Mapping=tex-text,Numbers=Lining]{Linux Libertine O}
% \setmonofont[Scale=MatchLowercase]{DejaVu Sans Mono}
% \usepackage[libertine]{newtxmath}

\title{CS 1340:Fall 2020:Lecture 02}
\subtitle{Intro to Python for CS and Data Science}
\author{Mark Fontenot, PhD}
\date{}
\institute{Southern Methodist University}

\begin{document}
\frame{\titlepage}

\begin{frame}{Reminders}
\protect\hypertarget{reminders}{}
\begin{itemize}
\tightlist
\item
  Slack
\item
  Zybooks (hopefully you did the assignment due before class today?)
\item
  Anaconda?
\end{itemize}
\end{frame}

\begin{frame}{All computer programs have}
\protect\hypertarget{all-computer-programs-have}{}
\begin{itemize}
\tightlist
\item
  Input
\item
  Processing
\item
  Output
\end{itemize}
\end{frame}

\begin{frame}{Test}
\protect\hypertarget{test}{}
\begin{enumerate}
\tightlist
\item
  item 1
\item
  item 2
\end{enumerate}
\end{frame}

\begin{frame}{Welcome}
\protect\hypertarget{welcome}{}
\begin{itemize}
\item
  About me\ldots{}

  Mark Fontenot, PhD

  Caruth 433

  mfonten@lyle.smu.edu

  markfontenot.net
\end{itemize}
\end{frame}

\begin{frame}{Bookmarks!}
\protect\hypertarget{bookmarks}{}
\textbf{Course Website}:
\url{https://smu-cs1340.github.io/cs1340-f2020/}

\textbf{Github}: \url{https://github.com} \emph{\ldots{} have you made
your account yet?}

\textbf{Canvas}, of course
\end{frame}

\begin{frame}[fragile]{Textbook}
\protect\hypertarget{textbook}{}
\begin{itemize}
\tightlist
\item
  Python ZyBooks

  \begin{itemize}
  \tightlist
  \item
    Interactive learning
  \item
    learn.zybooks.com
  \item
    ZyBook code for us: \texttt{SMUCS1340FontenotFall2020}
  \item
    Pay and subscribe
  \end{itemize}
\item
  (Optional) Eric Matthes. \emph{Python Crash Course, 2nd edition}. No
  Starch Press, 2019, ISBN-10: 1-59327-928-0
\end{itemize}
\end{frame}

\begin{frame}{Evaluation}
\protect\hypertarget{evaluation}{}
\begin{itemize}
\tightlist
\item
  Final grades determined using standard 10pt scale.
\end{itemize}

\begin{longtable}[]{@{}lr@{}}
\toprule
Tool & Percentage\tabularnewline
\midrule
\endhead
Homeworks and Quizzes & 25\%\tabularnewline
Programming Projects & 35\%\tabularnewline
Final Project & 15\%\tabularnewline
Final Exam & 15\%\tabularnewline
Attendance \& Participation & 10\%\tabularnewline
\bottomrule
\end{longtable}
\end{frame}

\begin{frame}{Homework and Quizzes}
\protect\hypertarget{homework-and-quizzes}{}
\begin{itemize}
\tightlist
\item
  Homework will be mostly related to ZyBooks Exercises
\item
  Quizzes maybe administered in lecture or lab.
\end{itemize}
\end{frame}

\begin{frame}{Programming Projects}
\protect\hypertarget{programming-projects}{}
\begin{itemize}
\tightlist
\item
  Learn by doing!
\item
  If you've never programmed before, don't worry! We will guide you!
\item
  10\% bonus credit for submitting 48 hours early.
\item
  Everyone gets one 2-day free extension, no questions asked.\\
\item
  You need to plan for at least a few hours of work per week outside of
  lecture and lab
\end{itemize}
\end{frame}

\begin{frame}{Attendance}
\protect\hypertarget{attendance}{}
\begin{itemize}
\item
  You'll need to attend class to be successful.
\item
  I'll talk about things that aren't necessarily in ZyBooks.
\item
  If you're attending virtually:

  \begin{itemize}
  \tightlist
  \item
    Turn your camera on and microphone off
  \item
    Please be mostly vertical (aka. not in bed)
  \item
    Please be dressed as you would if attending class in person.
  \end{itemize}
\end{itemize}
\end{frame}

\begin{frame}{Academic Ethics}
\protect\hypertarget{academic-ethics}{}
\begin{itemize}
\tightlist
\item
  You must be prepared to defend any code or assignment that you
  submit.\\
\item
  DO NOT ever share your code with anyone. DO NOT, DO NOT, DO NOT!!!
\item
  You are encouraged to talk about the assignments at an abstract level.
  However, you cannot share your solutions with another student.
\end{itemize}
\end{frame}

\begin{frame}{General Comments}
\protect\hypertarget{general-comments}{}
\begin{itemize}
\tightlist
\item
  Wear a face mask at all times
\item
  Try and keep your phone from dominating your attention.
\item
  Don't surf r/memes or other sites that may be distracting to those
  sitting around you
\item
  Ostentatious yawning is a huge pet peeve\ldots{}
\end{itemize}
\end{frame}

\hypertarget{the-course}{%
\section{the course}\label{the-course}}

\begin{frame}{Apps}
\protect\hypertarget{apps}{}
\begin{quote}
Besides the boring and obvious \ldots{}
\end{quote}

What's your favorite app? OR What app do you use the most?
\end{frame}

\begin{frame}{Question}
\protect\hypertarget{question}{}
Have you ever coded before?

\note{ 
    What does it mean to code? 
    What's an algorithm? 
}
\end{frame}

\begin{frame}{What are we going to cover?}
\protect\hypertarget{what-are-we-going-to-cover}{}
\begin{itemize}
\tightlist
\item
  Computational thinking (huh?)
\item
  The Python Programming Language
\item
  Basic Python for Data Science Applications
\end{itemize}
\end{frame}

\begin{frame}{Algorithm}
\protect\hypertarget{algorithm}{}
\begin{quote}
What's an Algorithm?
\end{quote}
\end{frame}

\begin{frame}{Computers Do Exactly What You Tell them To}
\protect\hypertarget{computers-do-exactly-what-you-tell-them-to}{}
\href{https://www.youtube.com/watch?v=Ct-lOOUqmyY}{The Exact Instruction
Challenge}
\end{frame}

\begin{frame}{A Program}
\protect\hypertarget{a-program}{}
\begin{itemize}
\tightlist
\item
  What's a computer program?
\end{itemize}
\end{frame}

\begin{frame}{What do I expect you to know?}
\protect\hypertarget{what-do-i-expect-you-to-know}{}
\begin{itemize}
\tightlist
\item
  You should know how to use your computer.
\item
  How to use the file browser on your computer.
\item
  How to Google!
\item
  How not to meltdown when something doesn't work :)
\end{itemize}
\end{frame}

\begin{frame}{Getting Set up for Python}
\protect\hypertarget{getting-set-up-for-python}{}
\textbf{Anaconda}

\begin{itemize}
\item
  Anaconda Python is a particular distribution of the language plus
  bunches of other tools
\item
  Makes it a snap to install on multiple platforms
\item
  Go to \url{https://www.anaconda.com/products/individual}, scroll to
  the bottom, and choose the installer for your platform.

  \begin{itemize}
  \tightlist
  \item
    More platform-specific instructions:
    \url{https://docs.anaconda.com/anaconda/install/}
  \end{itemize}
\item
  Open Anaconda Navigator to verify it was installed.
\end{itemize}
\end{frame}

\begin{frame}{To Do:}
\protect\hypertarget{to-do}{}
\begin{itemize}
\tightlist
\item
  Register for Slack and the \textbf{\#20f-1340-general} channel.

  \begin{itemize}
  \tightlist
  \item
    Add a profile picture!
  \end{itemize}
\item
  Install Anaconda!
\item
  Read Chapter 1 of ZyBooks before Thursday's class. \emph{(I'll send
  out some more specific directions on the various exercises tomorrow.)}
\end{itemize}
\end{frame}

\end{document}
